\documentclass[12pt,a4paper]{report}


\usepackage[utf8]{inputenc}
\usepackage{amsmath}
\usepackage{amsfonts}
\usepackage{amssymb}
\usepackage{graphicx}
\usepackage{amsthm}
\usepackage{natbib}
\usepackage{algorithm}
\usepackage{algpseudocode}

\usepackage{hyperref}
\AtBeginDocument{\let\textlabel\label}

\usepackage{marginnote}
\renewcommand*{\marginfont}{\scriptsize }







\theoremstyle{plain}
\newtheorem{thm}{Theorem}
\newtheorem{lemma}{Lemma}
\newtheorem{prop}{Proposition}

\theoremstyle{definition}
\newtheorem{definition}{Definition}
\newtheorem{remark}{Remark}
\newtheorem{example}{Example}


% Custom commands

\newcommand{\naive}{na\"{\i}ve }
\newcommand{\Naive}{Na\"{\i}ve }
\newcommand{\andor}{and\textbackslash or }

\newcommand{\al}{\alpha}
\newcommand{\be}{\beta}

\newcommand{\set}[1]{\left\{ #1 \right\}} % A set
\newcommand{\rv}[1]{\mathbf{#1}} % A random variable
\newcommand{\x}{\rv x} % The random variable x 
\newcommand{\y}{\rv y} % The random variable y
\newcommand{\expect}[1]{\mathbf{E}\left[ #1 \right]} % The expectation operator
\newcommand{\expectg}[2]{\mathbf{E}_{\rv{#1}}\left[ \rv{#2} \right]} % An expectation w.r.t. a particular random variable.
\newcommand{\expectn}[1]{\mathbb{E}\left[#1\right]} % The empirical expectation
\newcommand{\cov}[1]{\mathbf{C}ov \left[ #1 \right]} % The expectation operator
\newcommand{\covn}[1]{\mathbb{C}ov \left[ #1 \right]} % The expectation operator
\newcommand{\gauss}[1]{\mathcal{N}\left(#1\right)} % The gaussian distribution
\newcommand{\cdf}[2]{F_\rv{#1} (#2)} % The CDF function
\newcommand{\cdfn}[2]{\mathbb{F}_{#1}(#2)} % The empirical CDF function
\newcommand{\icdf}[2]{F_\rv{#1}^{-1} (#2)} % The invecrse CDF function
\newcommand{\icdfn}[2]{\mathbb{F}^{-1}_{#1}(#2)} % The inverse empirical CDF function
\newcommand{\pdf}{p} % The probability density function
\newcommand{\dist}{P} % The proabaiblity distribution

\newcommand{\estim}[1]{\widehat{#1}} % An estimator

\newcommand{\norm}[1]{\Vert #1 \Vert} % The norm operator
\newcommand{\normII}[1]{\norm{#1}_2} % The norm operator
\newcommand{\normI}[1]{\norm{#1}_1} % The norm operator
\newcommand{\lik}{\mathcal{L}} % The likelihood function
\newcommand{\loglik}{L} % The log likelihood function
\newcommand{\loss}{l} % A loss function
\newcommand{\risk}{R} % The risk function
\newcommand{\riskn}{\mathbb{R}} % The empirical risk
\newcommand{\deriv}[2]{\frac{\partial #1}{\partial #2}} % A derivative
\newcommand{\argmin}[2]{argmin_{#1}\set{#2}} % The argmin operator
\newcommand{\argmax}[2]{argmax_{#1}\set{#2}} % The argmin operator
\newcommand{\hyp}{f} % A hypothesis
\newcommand{\hypclass}{\mathcal{F}} % A hypothesis class
\newcommand{\hilbert}{\mathcal{H}}
\newcommand{\rkhs}{\hilbert_\kernel} % A hypothesis class
\newcommand{\normrkhs}[1]{\norm{#1}_{\rkhs}}

\newcommand{\plane}{\mathbb{L}} % A hypoerplane
\newcommand{\categories}{\mathcal{G}} % The categories set.
\newcommand{\positive}[1]{\left[ #1 \right]_+} % The positive part function
\newcommand{\kernel}{\mathcal{K}} % A kernel function
\newcommand{\featureS}{\mathcal{X}} % The feature space
\newcommand{\indicator}[1]{I_{\set{#1}}} % The indicator function.
\newcommand{\reals}{\mathbb{R}}



\newcommand{\latent}{S}
\newcommand{\loadings}{A}
\newcommand{\rotation}{R}
\newcommand{\similaritys}{\mathfrak{S}}
\newcommand{\similarity}{s} % A similarity measure.
\newcommand{\dissimilarity}{d} % A similarity measure.
\newcommand{\dissimilaritys}{\mathfrak{D}}



\newcommand{\manifold}{\mathcal{M}} % A manifold.
\newcommand{\project}{\hookrightarrow} % The orthogonal projection operator.
\newcommand{\projectMat}{H} % A projection matrix.
\newcommand{\rank}{q} % A subspace rank.
\newcommand{\encode}{E}
\newcommand{\decode}{D}


\newcommand{\ensembleSize}{M} % Size of a hypothesis ensemble.
\newcommand{\ensembleInd}{m} % Index of a hypothesis in an ensemble.


\newcommand{\sample}{\mathcal{S}} % A data sample.
\newcommand{\test}{\risk(\hyp)} % The test error (risk)
\newcommand{\train}{\riskn(\hyp)} % The train error (empirical risk)
\newcommand{\insample}{\bar{\risk}(\hyp)} % The in-sample test error.
\newcommand{\EPE}{\risk(\hat{\hyp}_n)} % The out-of-sample test error.
\newcommand{\folds}{K} % Cross validation folds 
\newcommand{\fold}{k} % Index of a fold
\newcommand{\bootstraps}{B} % Bootstrap samples
\newcommand{\bootstrap}{{b^*}} % Index of a bootstrap replication
