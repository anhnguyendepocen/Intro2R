
% % % % % % RKHS % % % % %

\chapter{The Kernel Trick and Reproducing Kernel Hilbert Spaces (RKHS)}
\label{apx:rkhs}

In the context of supervised learning the \emph{kernel trick} is a mathematical device that allows to learn very complicated predictors ($\hyp$) in a computationally efficient manner. 
More generally, in the context of unsupervised learning, the kernel tricks allow to learn complicated non-linear mappings of the original features (and not only predictor functions). 

Not all predictors and not all problem admit this trick. Then again, many do. 
Methods for which it applies include:
SVM's (\S\ref{sec:svm}), principal components analysis (\S\ref{sec:pca}), canonical correlation analysis (\S\ref{sec:cca}), ridge regression (\S\ref{sec:ridge}), spectral clustering (\S\ref{sec:spectral_clustering}), Gaussian processes\footnote{See the Bayesian interpretation below to see why they apply to Gaussian Processes.}, and more\footnote{This partial list is taken from Wikipedia: \url{http://en.wikipedia.org/wiki/Kernel_method}}.

We now give an exposition of the method in the context of supervised learning. 


Think of smoothing splines (\S\ref{sec:smoothing_splines});
It was quite magical that without constraining the hypothesis class $\hypclass$, the ERM problem in Eq.(\ref{eq:smoothing_spline}) has a finite dimensional closed form solution.
The property of an infinite dimensional problem having a solution in a finite dimensional space is known as the \emph{kernel property}.\marginnote{Kernel Property}
We with to generalize this observation and ask- which problems have the kernel property?
Stating the general optimization problem:
\begin{align}
\label{eq:rkhs}
	\argmin{\hyp}{\frac{1}{n} \sum_i \loss(y_i,\hyp(x_i)) + \lambda J(\hyp) }
\end{align}
The problem is then- what type of penalties $J(\hyp)$ will return simple solutions to Eq.(\ref{eq:rkhs}).
The answer is: function that belong to \emph{Reproducing Kernel Hilbert Space} function spaces. 
RKHS's are denoted by $\rkhs$.
They include many functions, but they are a rather ``small'' subset of the space of all possible functions.
These spaces, and the functions therein, are defined by another function called a \emph{Kernel} denoted by $\kernel$. 
Choosing a particular kernel defines the space and the functions therein.
Choosing a particular kernel, also defines the form of $J$ in Eq.(\ref{eq:rkhs}).
Put differently: for any choice of a kernel $\kernel$, there is a particular $J(\hyp)$ for which the solution of Eq.(\ref{eq:rkhs}) will be a function in $\rkhs$ and will be easily computable. 


\section{Mathematics of RKHS}
We now show how choosing a kernel $\kernel$ defines a space $\rkhs$, and a penalty $J(\hyp)$.

A kernel is a non-negative symmetric function of two arguments: $\kernel(x,y): \reals^p \times \reals^p \mapsto \reals_+$.
By fixing $y$, $\kernel(x,y)$ is a function with a single argument $x \mapsto \kernel(x,y)$.
$\rkhs$ is merely the space of functions of $x$, spanned at given $y$'s: 
\begin{align}
\label{eq:rkhs_span}
	\hyp(x):\sum_m \al_m \kernel(x,y_m)
\end{align} 

From linear algebra, you may know that positive definite matrices be diagonalized. 
This analogy carries to $\kernel$, which admits an eigen-expansion: 
\begin{align}
\label{eq:rkhs_eigen}
	\kernel(x,y)=\sum_{i=1}^\infty \gamma_i \phi(x) \phi(y)
\end{align}
Using Eqs.(\ref{eq:rkhs_eigen}) and (\ref{eq:rkhs_span}) we can thus expand elements $f$ of $\rkhs$: 
\begin{align}
	\hyp(x)=\sum_{i=1}^\infty c_i \phi(x)
\end{align}
where $c_i=\gamma_i \sum_m \alpha_m \phi(y)$.
We also define a norm $\normrkhs{\hyp}^2$ in this space, which is induced by $\kernel$:
\begin{align}
\label{eq:rkhs_norm}
	\normrkhs{\hyp}^2 := \sum_{i=1}^\infty \frac{c_i^2}{\gamma_i}
\end{align}

The penalty $J(\hyp)$ in Eq.(\ref{eq:rkhs}), is simply be $\normrkhs{\hyp}^2$.
The $\hyp$'s that solve Eq.(\ref{eq:rkhs}) are guaranteed to have a simple form. They reside in an $n$ dimensional linear function space \citep{wahba_spline_1990}:
\begin{align}
	\hyp(x)=\sum_{i=1}^n \al_i \kernel(x,x_i)
\end{align}

The functions $\kernel(x,x_i)$ can be seen as a basis to the solution space. 
The good news continue! Being only $n$ dimensional, the norms of these $\hyp$'s, do not require integration but rather only finite summation:
\begin{align}
	\normrkhs{\hyp}^2=\sum_{i=1}^n \sum_{j=1}^n \kernel(x_i,x_j) \al_i \al_{j} := \al' K \al.
\end{align}

Adding the above results, we can restate Eq.(\ref{eq:rkhs}) and say that when fixing $\kernel$ and using the appropriate $J$, we only need to solve: 
\begin{align}	
\label{eq:rkhs_simple}
		\argmin{\al}{\frac{1}{n} \sum_i \loss(y_i-K_i \alpha) + \lambda \al' K \al }
\end{align}
which is a quadratic programming problem over an $n$ dimensional linear space, easily solvable with numeric routines.


\subsection{The Bayesian View of RKHS}
Just as the ridge regression (\S\ref{sec:ridge}) has a Bayesian interpretation, so does the kernel trick.
Informally, the functions solving Eq.(\ref{eq:rkhs}) can be seen as the posterior mode if our prior beliefs postulate that the function we are trying to recover is a Gaussian zero-mean process with covariance given by $\kernel$.
This view suggests the intuition that the regularization introduced by $J(\hyp)$ shrinks the estimated $\hyp$ towards a smoother function. At an extreme, where $\lambda\to\infty$, we will recover a constant function, since the the mode of our Gaussian process prior is at the origin of $\rkhs$.




% % % % % % % % Generative models % % % % % % % %

\chapter{Generative Models}
\label{apx:generative_concept}

By \emph{generative model} we mean that we specify the whole data distribution. This is particularly relevant to supervised learning where many methods only assume the distribution of $\dist(y|x)$ without stating the distribution of $\dist(x)$.
Assuming only $\dist(y|x)$ is known as a \emph{discriminative model}, or \emph{discriminative analysis}.\marginnote{Descriminative Model}
In a generative model, in contrast, we assume the whole $\dist(y,x)$.

For the mere purpose of making a prediction, we do not need to learn $\dist(y,x)$. 
Knowing this distribution, however, does permit to make predictions, via Bayes Theorem: 
$\dist(y|x)=\frac{\dist(y,x)}{\int\dist(y,x)dy}$.
Generative models make use of this relation to make predictions. 

To gain some intuition, consider a supervised learning problem where the data has an equal number of samples per class. 
Learning the distribution of $x$ withing each class, allows to a simple classification of a given $x$ to the class with highest probability. LDA (\S\ref{sec:lda}), QDA (\S\ref{sec:lda}), and \Naive Bayes (\S\ref{sec:naive_bayes}) follow this exact same rational.




% % % % % % % % % Dimensionality Reduction % % % % % % %

\chapter{Dimensionality Reduction}
\label{apx:dim_reduce}

Dimensionality reduction is a useful concept in both supervised and unsupervised learning. 

The fundamental idea behind dimensionality reduction is that while $\featureS$ may be high dimensional, thus $\dist(x)$ hard to learn, there is hope that $\x$ does not really vary in the whole space. 
If the mass of $\dist(x)$ is concentrated around some low dimensional manifold $\manifold$, then the original problem might be approximated to learning the distribution of the projection $\dist(X \project \manifold)$ on $\manifold$. 
If $\manifold$ is fairly low dimensional, we may hope to visualize and understand $\dist(X \project \manifold)$ with fairly simple tools.
Dimensionality reduction also reduces the memory required to represent the data. It is thus intimately related to \emph{lossy compression} in information theory.\marginnote{Lossy Compression}

A similar reasoning justifies dimensionality reduction in supervised learning. 
While $\dist(x)$ might vary in the whole $\featureS$, but there might be only few directions which carry information on $y$. Learning $\dist(y|x)$ can thus be well approximated by $\dist(y|x \project \manifold)$.


\begin{remark}
The subspace $\manifold$, which approximates $\dist(x)$ may differ than the one that approximates $\dist(y|x)$. Despite this, it is still quite common for a supervised learning problem to be preceded by an unsupervised dimensionality reduction stage.
\end{remark}


% % % % % % Notation % % % % %


\chapter{Notation}
\label{apx:notation}

In this text we use the following notation conventions:
\begin{description}
\item[$x$] A scalar or vector.
\item[$\x$] A (possibly) vector valued random variable.
\item[$X$] A matrix.
\item[$\x$]  A matrix valued random variable (a random matrix).
\item[$X'$] The matrix transpose of $X$.
\item[$\normII{x}$] The $l_2$ norm of $x$: $\sqrt{\sum_j x_j^2}$.
\item[$\normI{x}$] The $l_1$ norm of $x$: $\sum_j |x_j|$
\item[$\sample$] A data sample.
\item[$\expect{\x}$] The expectation of $\x$.
\item[$\expectn{x}$] The empirical expectation (average) of the vector $x$.
\item[$\cov{\x}$] The covariance matrix of $\x$: $\expect{(\x-\expect{\x})(\x-\expect{\x})'}$.
\item[$\covn{x}$] The empirical covariance matrix of x: $\expectn{(x-\expectn{x})(x-\expectn{x})'}$.
\item[$\cdf{x}{t}$] The CDF of $\x$ at $t$.
\item[$\icdf{x}{\al}$] The inverse CDF at $\al$ (the quantile function).
\item[$\cdfn{x}{t}$] The empirical CDF of data vector $x$.
\item[$\icdfn{x}{\al}$] The empirical $\al$ quantile of the data vector $x$.
\item[$\x \sim \dist$]  The random variable $\x$ is $\dist$ distributed. 
\item[$\pdf(x)$] The density function of $\dist$ at $x$.
\item[$\gauss{\mu,\sigma^2}$] The univariate Gaussian distribution with mean $\mu$ and variance $\sigma^2$.
\item[$\gauss{\mu,\Sigma}$] The multivariate Gaussian distribution with mean vector $\mu$ and covariance matrix $\Sigma$.
\item[$\lik(\theta)$] The likelihood function at $\theta$.
\item[$\loglik(\theta)$] The log likelihood function at $\theta$.
\item[$\loss(x,\theta)$] The loss function of $\theta$ at $x$.
\item[$\risk(\theta)$]  The risk at $\theta$.
\item[$\riskn(\theta)$] The empirical risk at $\theta$.
\item[$\hyp(x)$] A prediction (hypothesis) at $x$.
\item[$\hypclass$] The class of all hypotheses $\hyp$.
\item[$\plane$] A hyperplane.
\item[$\categories$] A set of categories. 
\item[$\positive{t}$] The positive part of $t$.
\item[$\kernel(x,y)$] A kernel function evaluated at $(x,y)$.
\item[$\indicator{A}$] The indicator function of the set $A$.
\item[$\manifold$] A manifold.
\item[$\project$] A projection operator.

\item[$\similarity_{ij}$] A similarity measure between observations $i$ and $j$. 
\item[$\dissimilarity_{ij}$] A dissimilarity measure between observations $i$ and $j$. 
\item[$\similaritys$] A weighted graph (network),i.e. matrix, of similarities between observations. 
\item[$\dissimilaritys$] A weighted graph (network),i.e. matrix, of dissimilarities between observations. 


\end{description}


